
\subsection{Being Positively Negative}
\label{sec:idioms:posneg}

A workhorse of dependently typed programming is using decidable predicates capture, and produce, positive information.
We can represent the results of a decidable function using \texttt{Dec}:

\begin{verbatim}
data Dec a = Yes a | No (a -> Void)
\end{verbatim}

\noindent
Notice that in the negative position (the \texttt{No} constructor) we do not return positive information.
Wee provide a proof of falsity.
When reasoning about our programs such proof of falsity is good.
When interacting with our programs such proof of falsity is not good.
Consider the following type signatures that specify two decidable procedures.
The first to determine if one natural number is greater than another.

\begin{verbatim}
isGT : (x,y : Nat) -> Dec (GT x y)
\end{verbatim}

\noindent
And the other, to determine if any element in the list satisfies a supplied predicate.

\begin{verbatim}
any : (f  : (x : type) -> Dec (p x)) -> (xs : List type) -> Dec (Any p xs)
\end{verbatim}

For \texttt{isGT} we know that if the result is negative we can safely assume that \texttt{GT x y} is false.
With \texttt{any} we do not know for sure why the decidable procedure failed.
Was it because the list was empty?
or,
was it because all elements of \texttt{xs} did not satisfy \texttt{p}!?
More so what was the reason for any element in \texttt{xs} for not satisfying \texttt{p}?

To address these issues. we introduce \texttt{DecInfo} a variant of \texttt{Dec} that carries positive information in the negative case, as well as proofs-of-falsity.

\begin{verbatim}
data DecInfo e p = Yes p | No e (p -> Void)
\end{verbatim}

With this simple change we can now report \emph{why} a decision procedure failed, as well as proof that it failed.
Our examples can now become:

\begin{verbatim}
isGT : (x,y : Nat) -> DecInfo (LTE x y) (GT x y)
\end{verbatim}

\noindent
and

\begin{verbatim}
any : (f  : (x : type) -> Dec (q x) (p x))
   -> (xs : List type)
         -> DecInfo (All (\x => Pair (q x) (Not (p x))) xs) (Any p xs)
\end{verbatim}

\texttt{DecInfo} is, however, not \emph{strictly} positive as it carries a proof of false.
A more interesting line of work would be to adopt \emph{constructive negation} to ensure that proofs of false arise from positive sources only~\cite{msfp/Atkey22}.
That is, given:

\begin{verbatim}
Predicate : Type
Predicate = (pos ** neg ** (pos -> neg -> Void))
\end{verbatim}

we can recreate \texttt{DecInfo} as:

\begin{verbatim}
DecInfo : Predicate -> Type
DecInfo (pos ** neg ** no) = Either neg pos
\end{verbatim}

This is line of investigation is on going.
