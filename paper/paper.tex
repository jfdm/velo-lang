\documentclass[%
draft, % note that the orcid.pdf logo is not included with draft mode
a4paper,
UKenglish,
cleveref,
autoref,
thm-restate,
%anonymous,
pdfa
]{oasics-v2021}

\usepackage{todonotes}
\setuptodonotes{inline}

\usepackage{relsize}

\usepackage{tikz}
\usetikzlibrary{shapes.geometric}

\usepackage{acronym}
\usepackage{xspace}
\usepackage[commandnameprefix=always]{changes}
\usepackage{fancyvrb}
\usepackage{jfdm-plt}
\usepackage{velo}

\pdfoutput=1
\hideOASIcs %uncomment to remove references to OASIcs series (logo, DOI, ...), e.g. when preparing a pre-final version to be uploaded to arXiv or another public repository

%\graphicspath{{./graphics/}}%helpful if your graphic files are in another directory

\bibliographystyle{plainurl}% the mandatory bibstyle

\title{Type Theory as a Language Workbench} %TODO Please add

\author
{Jan {de Muijnck-Hughes}}
{University of Glasgow, UK}
{Jan.deMuijnck-Hughes@glasgow.ac.uk}
{https://orcid.org/0000-0003-2185-8543}
{} % TODO

\author
{Guillaume Allais}
{University of St Andrews, UK}
{gxa1@st-andrews.ac.uk} % Check
{https://orcid.org/0000-0002-4091-657X} % ORCID
{} % TODO

\author
{Edwin Brady}
{University of St Andrews, UK}
{ecb10@st-andrews.ac.uk} % Check
{} % ORCID
{} % TODO

% TODO mandatory, please use full name; only 1 author per \author macro; first two parameters are mandatory, other parameters can be empty. Please provide at least the name of the affiliation and the country. The full address is optional. Use additional curly braces to indicate the correct name splitting when the last name consists of multiple name parts.

\authorrunning{J. {de Muijnck-Hughes}, G. Allais, and E. Brady}
% TODO mandatory. First: Use abbreviated first/middle names. Second (only in severe cases): Use first author plus 'et al.'

\Copyright{Jan de Muijnck-Hughes, Guillaume Allais, and Edwin Brady}
% TODO mandatory, please use full first names. LIPIcs license is "CC-BY";  http://creativecommons.org/licenses/by/3.0/

%\begin{CCSXML}
%<ccs2012>
%   <concept>
%       <concept_id>10003752.10003790.10011740</concept_id>
%       <concept_desc>Theory of computation~Type theory</concept_desc>
%       <concept_significance>500</concept_significance>
%       </concept>
%   <concept>
%       <concept_id>10011007.10010940.10010992.10010998.10010999</concept_id>
%       <concept_desc>Software and its engineering~Software verification</concept_desc>
%       <concept_significance>500</concept_significance>
%       </concept>
%   <concept>
%       <concept_id>10011007.10011006.10011008.10011009.10011012</concept_id>
%       <concept_desc>Software and its engineering~Functional languages</concept_desc>
%       <concept_significance>500</concept_significance>
%       </concept>
%   <concept>
%       <concept_id>10011007.10011006.10011039</concept_id>
%       <concept_desc>Software and its engineering~Formal language definitions</concept_desc>
%       <concept_significance>500</concept_significance>
%       </concept>
%   <concept>
%       <concept_id>10011007.10011006.10011050.10011017</concept_id>
%       <concept_desc>Software and its engineering~Domain specific languages</concept_desc>
%       <concept_significance>500</concept_significance>
%       </concept>
%   <concept>
%       <concept_id>10002950.10003714.10003732.10003733</concept_id>
%       <concept_desc>Mathematics of computing~Lambda calculus</concept_desc>
%       <concept_significance>500</concept_significance>
%       </concept>
% </ccs2012>
%\end{CCSXML}

\ccsdesc[500]{Theory of computation~Type theory}
\ccsdesc[500]{Software and its engineering~Software verification}
\ccsdesc[500]{Software and its engineering~Functional languages}
\ccsdesc[500]{Software and its engineering~Formal language definitions}
\ccsdesc[500]{Software and its engineering~Domain specific languages}
\ccsdesc[500]{Theory of computation~Lambda calculus}
\ccsdesc{Software and its engineering~Compilers}
\ccsdesc{Theory of computation~Invariants}

%\ccsdesc[100]{\textcolor{red}{Replace ccsdesc macro with valid one}}
% TODO mandatory: Please choose ACM 2012 classifications from https://dl.acm.org/ccs/ccs_flat.cfm

\keywords{dependent types, language workbenches, idris2, dsl, edsl, intrinsically scoped, well typed, co-De~Bruijn}
% TODO mandatory; please add comma-separated list of keywords

\category{Research}
% optional, e.g. invited paper

\relatedversion{}

\supplement{We have made V{\'e}lo's source available both as a reproducible artifact and freely available source code:}
\supplementdetails[
subcategory={Source Code}
]{Software}{https://github.com/jfdm/velo-lang}
\supplementdetails[
subcategory={Artifact}
]{Software}{https://doi.org/10.5281/zenodo.7573031}
% \supplement{}
% optional, e.g. related research data, source code, ... hosted on a repository like zenodo, figshare, GitHub, ...
% \supplementdetails[linktext={opt. text shown instead of the URL}, cite=DBLP:books/mk/GrayR93, subcategory={Description, Subcategory}, swhid={Software Heritage Identifier}]{General Classification (e.g. Software, Dataset, Model, ...)}{URL to related version}
% linktext, cite, and subcategory are optional

% \funding{(Optional) general funding statement \dots}
% optional, to capture a funding statement, which applies to all authors. Please enter author specific funding statements as fifth argument of the \author macro.

% \acknowledgements{I want to thank \dots}
% optional

\nolinenumbers %uncomment to disable line numbering

%Editor-only macros:: begin (do not touch as author)%%%%%%%%%%%%%%%%%%%%%%%%%%%%%%%%%%
\EventEditors{Ralf L\"{a}mmel, Peter D. Mosses, and Friedrich Steimann}
\EventNoEds{3}
\EventLongTitle{Eelco Visser Commemorative Symposium (EVCS 2023)}
\EventShortTitle{EVCS 2023}
\EventAcronym{EVCS}
\EventYear{2023}
\EventDate{April 5, 2023}
\EventLocation{Delft, The Netherlands}
\EventLogo{}
\SeriesVolume{109}
\ArticleNo{22}
%%%%%%%%%%%%%%%%%%%%%%%%%%%%%%%%%%%%%%%%%%%%%%%%%%%%%%

\acrodef{dsl}[DSL]{Domain Specific Language}
\acrodef{edsl}[EDSL]{Embedded Domain Specific Language}
\acrodef{stlc}[STLC]{Simply-Typed Lambda Calculus}
\acrodef{cse}[CSE]{Common Sub-Expression Elimination}
\acrodef{ir}[IR]{Intermediate Representation}
\acrodef{repl}[REPL]{Read Eval Print Loop}
\acrodef{lsp}[LSP]{Language Server Protocol}

% Acronym & Cleverref clash is a subtle way :-(
% Ideal would be to switch to glossaries, which is my main driver, I wanted to be lightweight this time around...
%
% https://tex.stackexchange.com/questions/71364/acronym-acresetall-cleveref-multiply-defined-labels
\makeatletter
\newcommand*{\org@overidelabel}{}
\let\org@overridelabel\@verridelabel
\@ifpackagelater{acronym}{2015/03/21}{% v1.41
  \renewcommand*{\@verridelabel}[1]{%
    \@bsphack
    \protected@write\@auxout{}{\string\AC@undonewlabel{#1@cref}}%
    \org@overridelabel{#1}%
    \@esphack
  }%
}{% older versions
  \renewcommand*{\@verridelabel}[1]{%
    \@bsphack
    \protected@write\@auxout{}{\string\undonewlabel{#1@cref}}%
    \org@overridelabel{#1}%
    \@esphack
  }%
}
\makeatother

\definechangesauthor[color=orange,name={Jan}]{jfdm}
\definechangesauthor[color=green, name={Guillaume}]{gallais}
\definechangesauthor[color=red, name={Edwin}]{edwin}

\newcommand{\jfdm}[1]{\chcomment[id=jfdm]{#1}}
\newcommand{\gallais}[1]{\chcomment[id=gallais]{#1}}

\newcommand{\Velo}{V{\'e}lo\xspace}
\newcommand{\Idris}{Idris~2}
\newcommand{\DeBruijn}{De~Bruijn}
% In Dutch surnames the rules for capitalisation are as follows:
% Tussenvogels ('between letters' i.e. de,van der, van, et cetera) are capitalised if they are not between a first and last name.

\fvset
{ xleftmargin=0.5\parindent
, commandchars=\\\{\}
  % , fontsize=\smaller
  % , frame=leftline
  % , framerule=0.5mm
  % , rulecolor=\color{gray}
  % , numbers = left, numbersep=2pt
}

% Options...
\DefineVerbatimEnvironment%
  {VerbatimInline}
  {Verbatim}
  {xleftmargin=0.5\parindent
  , frame=leftline
  }
\DefineVerbatimEnvironment%
  {VerbatimInlineNumbered}
  {Verbatim}
  {xleftmargin=\parindent
    ,fontsize=\smaller
    ,numbers=left,numbersep=2pt
    ,frame=lines
  }

  % Does not work for Verbatim things, or grahpics.
\NewEnviron{centertight}[1][0.5em]
{\vspace{#1}
 {\centering\BODY}
 \vspace{#1}
}
%%% Local Variables:
%%% mode: latex
%%% TeX-master: "paper"
%%% End:


\begin{document}

\input{./notations}

\maketitle

% 1. State the problem
% 2. Say why it is an interesting problem
% 3. Say what your solution achieves
% 4. Say what follows from your solution

\begin{abstract}
We are planning to showcase dependently typed techniques to implement \& manipulate IRs.

The key components we are planning to treat are:

\begin{itemize}
\item Efficient de Bruijn representations
\item Co de Bruijn for CSE
\item Evaluation as Progress (with computation rules in one separate function)
\item Well scoped holes
\item Linear (in number of cases) decidable equality
\item Compact constant folding
\item Positive evidence for negative statements
\item Whole pipeline (parser, elaborator, REPL, evaluator, compiler passes)
\end{itemize}

Our ongoing work is available at:

\url{https://github.com/jfdm/velo-lang}
\end{abstract}

\todo{Run spellcheck}
\todo{Use Katla to typeset Idris 2 code}

\section{Introduction}
\label{sec:introduction}

\emph{Language Workbenches}, such as
Spoofax~\cite{DBLP:journals/software/WachsmuthKV14},
offer language designers an expressive environment in which to design,
implement, and deploy their \Acp{dsl}~\cite{hudak1996building}.
%
Principally speaking a language workbench~\cite{DBLP:conf/sle/ErdwegSVBBCGHKLKMPPSSSVVVWW13}
is a tool that supports:
description of a language's \emph{notation}---how we present a language's concrete syntax to users;
implementation of a language's \emph{semantics}---how we realise the language's behaviour;
and user interaction through an \emph{editor}.
%
Outside of these core criteria, various language workbenches
support language validation, testing, and composition.


Concurrently, the mechanised meta-theory research
programme~\cite{DBLP:conf/tphol/AydemirBFFPSVWWZ05,DBLP:journals/jfp/AbelAHPMSS19}
has seen a wealth of tools and techniques being developed
by the programming languages theory community.
%
In particular, dependently typed languages such as
\Idris{}~\cite{DBLP:conf/ecoop/Brady21},
Agda~\cite{DBLP:conf/afp/Norell08},
and Coq~\cite{the_coq_development_team_2022_5846982}
have been widely used to formalise \acp{dsl}, and study their
meta-theoretical properties.
%
Dependent types allow types to depend on values --- that is,
types are first class --- and provide an expressive environment
in which to reason about, and write, our programs.
%
Efforts using dependently typed languages range from
studying specific core calculi~\cite{10.1145/3093333.3009866,DBLP:conf/cpp/RouvoetPKV20,DBLP:conf/mpc/ChapmanKNW19}
to building generic reasoning frameworks~\cite{DBLP:conf/cpp/StarkSK19,DBLP:journals/jfp/AllaisACMM21}.
%
These mechanised software verification projects, however, typically stop short
of building the frontend that would let users run these verified
language implementations.
If our verified language implementations type check, we might as well ship them too!
%
By becoming its own implementation language, \Idris{} has successfully
demonstrated that this is not an inescapable fate~\cite{DBLP:conf/ecoop/Brady21}.
%
But can we now claim that dependently typed languages qualify as
language workbenches?

\Velo{}\footnote{\url{https://github.com/jfdm/velo-lang}}\footnote{For the final version we will replace this link to a DOI-backed artifact.} is a minimal functional language that we have realised in \Idris{}
to showcase dependently typed techniques to implement and manipulate \acp{ir}.
%
This paper introduces \Velo{} but, most of all, seeks to show that
dependently typed languages make fine language workbenches.
%
We address both the core criteria and some optional extensions
highlighted by the language workbench challenge~\cite{DBLP:conf/sle/ErdwegSVBBCGHKLKMPPSSSVVVWW13} for what constitutes a language workbench.
%
%
Although not all of the optional criteria
%for a language workbench
are met by dependently typed languages, we are convinced that
with some additional engineering (taking advantage of existing work,
for example Quickchick~\cite{DBLP:journals/pacmpl/LampropoulosPP18})
more optional criteria can be satisfied.

Another key tenet in language workbenches, such as Spoofax,
is the \emph{ease} with which languages can be created.
%
To that same degree, we have developed a series
of reusable modules
%, where possible,
that captures
functionality common to many languages,
%
thereby reducing the \emph{boilerplate} required when creating \acp{edsl} in \Idris{}.

Although we have made an effort to make dependently typed
programming accessible in our presentation, more introductory
material is available for the interested reader~\cite{plfa22.08,brady17:_type_driven_devel_idris}.

\todo{Add link to DOI-backed artifact}

%%% Local Variables:
%%% mode: latex
%%% TeX-master: "../paper"
%%% End:



\section{Introducing Velo}
\label{sec:velo}

The design behind \Velo{} is purposefully unsurprising:
%
it is \emph{just} the \ac{stlc} extended with let-bindings, booleans and their conjunction, and natural numbers and their addition.
%
To promote the idea of interactive editing \Velo{} also supports well-typed holes.
%
Below we show an example \Velo{} program, which contains a multiply used hole, and an extract from the \acs*{repl}
session that lists the current set of holes.

\begin{center}
  \begin{minipage}[t]{0.55\linewidth}
\begin{Verbatim}
   let b = false
in let double
         = (fun x : nat => (add x x))
in let x = (double ?hole)
in         (double ?hole)
\end{Verbatim}
\end{minipage}
\hfill
  \begin{minipage}[t]{0.35\linewidth}
    \begin{Verbatim}
Velo> :holes
b : Bool
double : Nat -> Nat
----------
?hole : Nat
\end{Verbatim}
\end{minipage}

\end{center}

The \emph{featherweight} language design of \Velo{} helps us showcase better \emph{how} we can use dependently typed languages as language workbenches~\cite{DBLP:journals/toplas/IgarashiPW01}.
%
Regardless of language complexity, \Velo{} is nonetheless a \emph{complete} language with a standard compiler pipeline, and \acs*{repl}.
%
A \ac{dsl} captures the language's concrete syntax, and a parser turns \ac{dsl} instances into raw unchecked terms.
%
Bidirectional type checking keeps type annotations to a minimun in the concrete syntax, and helps to better elaborate raw un-typed terms into a set of well-typed \acp{ir}:
%
\IdrisType{Holey} to support well-scoped typed holes;
%
and
%
\IdrisType{Terms} the core representation that captures our language's abstract syntax.
%
We present interesting aspects of our \ac{ir} design in \Cref{sec:design}.
%
Further, elaboration performs standard syntax transformations that turns let-bindings into function application thus reducing the size of our core.
%
From the core representation we also provide well-scoped \ac{cse} using co-\DeBruijn{} indexing (\Cref{sec:compiler-pass}), and we provide a verified evaluator to reduce terms to values (\Cref{sec:semantics}).


\todo{Show an example high-level trace of the output?}
\jfdm{This would be useful for a full-length paper, but not here.}

%%% Local Variables:
%%% mode: latex
%%% TeX-master: "../paper"
%%% End:


\section{Language Design}
\label{sec:design}

This section details key design rationale on realising the static and dynamic semantics of \Velo{} within Idris.

We have opted to give \Velo{} an external concrete syntax (a \ac{dsl}) in which users can write their programs.
%
With dependently typed languages, however, we can also capture the abstract syntax and its static semantics as a well-typed \ac{edsl} directly with the host language~\cite{Augustsson1999edt}.
Our terms will thus be implemented as a data structure with (almost) the following type:

\begin{Verbatim}
data Ty = TyNat | TyBool | TyArr Ty Ty
data Term : (ctxt : List Ty) -> (type : Ty) -> Type where
  Zero : Term ctxt TyNat
  Inc : Term ctxt TyNat -> Term ctxt TyNat
  -- ...
\end{Verbatim}

\noindent

Using such intrinsically typed \acp{edsl} we can statically verify for free that our language's are well-structured and that any transformation (model-to-model) or interpretation (model-to-host) respects the language's static semantics.
In fact we will describe in \Cref{sec:compiler} how we can use our \acp{edsl} to both verify our static semantics and describe our dynamic semantics.

\subsection{Efficient \DeBruijn{} Representation}
\label{sec:design:deBruijn}

A common strategy for implementing well-scoped terms is to use typed
\emph{\DeBruijn{}} indices~\cite{MANUAL:journals/math/debruijn72},
which are easily realised as an inductive family~\cite{DBLP:journals/fac/Dybjer94}
indicating where in the type-level context the variable is bound.

Concretely, we index the \IdrisType{Elem} family by a context
(once again represented as a \IdrisType{SnocList} of kinds) and
the kind of the variable it represents.

\begin{center}
\begin{minipage}{0.10\textwidth}
\varRule
\end{minipage}\hfill
\begin{minipage}{0.80\textwidth}
\ExecuteMetaData[Code/MiniVelo.idr.tex]{ElemDecl}
\end{minipage}
\end{center}

We then match each context membership inference rule to a constructor.
%
The \IdrisData{Here} constructor indicates that the variable of interest is
the most local one in scope (note the non-linear occurence of (\ty{x}{a}) on
the left hand side, and correspondingly of \IdrisBound{ty} on the right).

\begin{center}
\begin{minipage}{0.35\textwidth}
  \varZero
\end{minipage}\hfill
\begin{minipage}{0.55\textwidth}
\ExecuteMetaData[Code/MiniVelo.idr.tex]{varZero}
\end{minipage}
\end{center}

The \IdrisData{There} constructor skips past the most local variable to look for the variable of interest deeper in the context.

\begin{center}
\begin{minipage}{0.35\textwidth}
  \varSuc
\end{minipage}\hfill
\begin{minipage}{0.55\textwidth}
\ExecuteMetaData[Code/MiniVelo.idr.tex]{varSuc}
\end{minipage}
\end{center}


Whilst a valid definition, this approach unfortunately does not scale to
large contexts:
%
every \IdrisType{Elem} proof is linear in the size of the \DeBruijn{}
index that it represents.
%
To improve the runtime efficiency of the representation we instead opt to
model \DeBruijn{} indices as natural numbers, which \Idris{} compiles to
GMP-style unbounded integers.
%
Further, we need to additionally define an \IdrisType{AtIndex} family to ensure that
all of the natural numbers we use correspond to valid indices.
%
We pointedly reuse the \IdrisType{Elem} names because these \IdrisData{Here}
and \IdrisData{There} constructors play exactly the same role.

\ExecuteMetaData[Code/MiniDeBruijn.idr.tex]{AtIndexDef}

\noindent
We then define a variable as the pairing of a natural number and an \emph{erased}
(as indicated by the \IdrisKeyword{0} annotation on the binding site for \IdrisBound{prf})
proof that the given natural number is indeed a valid \DeBruijn{} index.

\ExecuteMetaData[Code/MiniDeBruijn.idr.tex]{IsVarDef}

Thanks to Quantitative Type Theory~\cite{DBLP:conf/birthday/McBride16,DBLP:conf/lics/Atkey18}
as implemented in \Idris{}, the compiler knows that it can safely erase these
runtime-irrelevant proofs.
%
we now have the best of both worlds: a well-scoped realisation of \DeBruijn{} indices
that is compiled efficiently.

\todo{Talk about smart constructors \& views?}
\jfdm{We should do this talking in the extended version...}

Just like the naïve definition of \DeBruijn{} indexing is not the
best suited for a practical implementation,
the inductive family \IdrisType{Term} described in Section~\ref{sec:design}
is not the most convenient to use.
%
We will now see one of its limitations and how we remedied it in
\Velo{}.

%%% Local Variables:
%%% mode: latex
%%% TeX-master: "../paper"
%%% End:

\subsection{Well-Typed Holes}
\label{sec:design:holes}

Holes are a special kind of placeholder that programmers can use for parts of
the program they have not yet written.
%
In a typed language, each hole will be assigned a type based on the context it
is used in.

Special language support (the ability to inspect, refine, compute with,
or fill an existing hole with an adequately typed term)
enables \emph{type-driven programming}~\cite{DBLP:journals/pacmpl/OmarVCH19},
a practice by which the user enters in a dialogue with the compiler in order
to interactively build the program.
%
Barebones language support should at least include the ability to inspect the
type of a hole and the local context it appears in, to instantiate it with an
adequately typed term, as well as the ability to
safely evaluate programs that still contain holes.
%
\Velo{} provides all three.
%
Our treatment of evaluation and instantiation are fairly standard, but our
elaboration process is more interesting.

\Idris{}'s elaborator lifts holes to top-level declarations with no associated
definition as it encounters them.
%
Because of this design choice users cannot mention the same hole explicitly in
different places to state their intention that these yet unwritten terms ought
to be the same.
%
They currently can refer to the hole's solution by its name, but that hole is
still very much placed in one specific position and it is from that position
that \Idris{} infers its context.

In \Velo{}, however, we allow holes to be mentioned arbitrarily many times in
arbitrarily different local contexts. In the following example, the hole
\texttt{?h} occurs in two distinct contexts: $\emptyset,a,x$ and $\emptyset,a,y$.

\begin{center}
  \holeexamplegraph{}
\end{center}

As a consequence, a term will only fit in that hole if it happens to live in the
shared common prefix of these two contexts ($\emptyset,a$).
%
Indeed, references to $x$ will not make sense in $\emptyset,a,y$ and vice-versa for $y$.


Our elaborator proceeds in two steps.
%
First, a bottom-up pass records holes as they are found and, in nodes
with multiple subterms, reconciles conflicting hole occurrences by
computing the appropriate local context restrictions.
%
This process produces a list of holes together with a \IdrisType{Holey}
term that contains invariants ensuring these collected holes do fit in the term.
%
Second, a top-down pass produces a core \IdrisType{Term} indexed by the list
of \IdrisType{Meta} (a simple record type containing the hole's name, the context
it lives in, and its type). Hole occurences end up being assigned a thinning
embedding the metavariable's actual context into the context it appears in.

Although these intermediate representations are \Velo{}-specific, the technique
itself is general and can be reused by anyone wanting to implement a well scoped
notion of holes.

%%% Local Variables:
%%% mode: latex
%%% TeX-master: "../paper"
%%% End:

\subsection{Compact Constant Folding}
\label{sec:design:constants}

Software Foundations' \emph{Programming Language Foundations}
opens with a constant-folding transformation exercise~\cite[Chapter~1]{Pierce:SF2}.
%
Starting from a small language of expressions (containing natural numbers, variables, addition, subtraction, and multiplication) we are to deploy the semiring properties to simplify expressions.
%
The definition of the simplifying traversal contains much duplicated code due to the way the source language is structured:
%
all the binary operations are separate constructors, whose subterms need to be structurally simplified before we can decide whether a rule applies.
%
The correction proof has just as much duplication because it needs to follow the structure of the call graph of the function it wants to see reduce.
%
The only saving grace here is that Coq's tactics language lets users write scripts that apply to many similar goals thus avoiding duplication in the source file.

In \Velo{}, we structured our core language's representation in an algebraic
manner so that this duplication is never needed.
%
All builtin operators (from primitive operations on builtin types to function
application itself) are represented using a single \IdrisData{Call} constructor
which takes an operation and a type-indexed list of subterms.


\ExecuteMetaData[Code/MiniCompact.idr.tex]{TermDef}

Here \IdrisType{Terms} is the pointwise lifting of \IdrisType{Term} to lists
of types. In practice we use the generic \IdrisType{All} quantifier, but this
is morally equivalent to the specialised version presented below:

\ExecuteMetaData[Code/MiniCompact.idr.tex]{AllDef}

The primitive operations can now be enumerated in a single datatype
\IdrisData{Prim} which lists the primitive operation's arguments and
the associated return type.

\begin{comment}
\IdrisData{Zero}---which takes no argument and returns a term of type \IdrisData{TyNat};
%
\IdrisData{Inc}---which takes an argument of type \IdrisData{TyNat} and returns a term
of type \IdrisData{TyNat};
%
and
%
\IdrisData{App}---which takes a function and an argument that corresponds to the type of the function's domain and returns a term that is the type of the function's co-domain.
\end{comment}

\ExecuteMetaData[Code/MiniCompact.idr.tex]{PrimDef}

Using \IdrisType{Prim}, structural operations can now be implemented by handling recursive calls on the subterms of \IdrisData{Call} nodes uniformly before dispatching on the operator to see whether additional simplifications can be deployed.
%
Similarly, all of the duplication in the correction proofs is factored out in a single place where the induction hypotheses can be invoked.


%%% Local Variables:
%%% mode: latex
%%% TeX-master: "../paper"
%%% End:



\section{Optimisation}
\label{sec:compiler-pass}
\chcomment{Co-\DeBruijn{} for CSE}

%TODO: comment on cost of thinning as inductive values?

Now that our core language is well scoped by construction, our compiler passes
must also be shown to be scope-preserving.
%
This is not a new requirement, merely it makes concrete a constraint that
used to be enforced informally.
More importantly we show, with our compiler passes, that model-to-same-model transformation of our \ac{edsl} is possible, and that the infrastructure required is not bespoke to \Velo{}.

The purpose of \ac{cse} is to identify subterms that appear multiple times in the syntax tree, and to abstract over them to avoid needless recomputations at runtime.
%
In the following example for instance, we would like to let-bind $t$ before
the application node (denoted \$) so that $t$ may be shared by both subtrees.

\begin{center}
  \cseexamplegraph{}
\end{center}

One of the challenges of \ac{cse}, as exemplified above, is that the term of interest
may be burried deep inside separate contexts.
%
In our intrinsically scoped representation, $t$ in scope $\Gamma, x : \sigma$
is potentially not actually syntactically equal to a copy living in $\Gamma, a : \tau, b : \nu$.
%
Indeed a variable $v$ bound in $\Gamma$ will, for instance, be represented by
the \DeBruijn{} index ($1+v$) in $\Gamma, x : \sigma$
but by the index ($2+v$) in $\Gamma, a :  \tau, b : \nu$.

If only terms were indexed by their exact \emph{support}!
We would not care about additional yet irrelevant variables that happen to be in scope.
%
The principled solution here is to switch to a different representation for
the purpose of \ac{cse}.
The co-\DeBruijn{} representation~\cite{DBLP:journals/corr/abs-1807-04085} provides exactly this guarantee.

%\subsection{Co-\DeBruijn{} representation}

In the co-\DeBruijn{} representation, every term is precisely indexed by its
exact support.
%
That is to say that every subterm explicitly throws away the bound variables
that are not mentioned in it.
%
By the time we reach a variable node, a single bound variable remains in scope:
precisely the one being referred to.

If we think of thinnings as sequences of 0/1 bits stating whether a variable
is kept or dropped, and admitting that such sequences can be represented as
list of either $\bullet$ (1) or $\circ$ (0), the $S$ combinator
($\lambda g. \lambda f. \lambda x. g x (f x)$) is represented as follows in
co-\DeBruijn{} notation (diagram taken from~\cite{MANUAL:draft/Allais22}).

\begin{center}
  \codebruijnexamplegraph{}
\end{center}

The first three $\lambda$ abstractions only use $\bullet$ in their thinnings
because all of $g$, $f$, and $x$ do appear in the body of the combinator.
%
The first application node then splits the context into two: the first subterm
($g x$) drops $f$ while the second ($f x$) gets rid of $g$.
%
Further application nodes select the one variable still in scope for each
leaf subterm: $g$, $x$, $f$, and $x$.

Using a co-\DeBruijn{} representation, we can easily identify shared subterms:
they need to not be mentioning any of the most local variables and be
syntactically equal.


Our pass succesfully transforms the program on the left-hand side to the
one on the right-hand side where the repeated expressions
\texttt{(add m n)} and \texttt{(add n m)} have been let-bound.

\begin{minipage}[t]{0.4\textwidth}
\begin{Verbatim}
let m = zero in
let n = (inc zero) in
(add (add (add m n) (add n m))
     (add (add n m) (add m n)))
\end{Verbatim}
\end{minipage}\hfill\begin{minipage}[t]{0.4\textwidth}
\begin{Verbatim}
let m = zero in
let n = (inc zero) in
let p = (add n m) in
let q = (add m n) in
(add (add q p) (add p q))
\end{Verbatim}
\end{minipage}


\section{Execution}
\label{sec:semantics}

The \Velo{} \acs*{repl} will let users reduce terms down to head-normal forms.
%
We can realise \Velo{}'s dynamic semantics either through definitional
interpreters~\cite{10.1145/3093333.3009866,Augustsson1999edt},
or by providing a more traditional syntactic proof of
type-soundness~\cite{DBLP:journals/iandc/WrightF94}
but mechanised~\cite[Part 2: Properties]{plfa22.08} using inductive families.

We chose the latter approach: by using inductive families, we can make explicit
our language's operational semantics.
%
This enables us to study its meta-theoretical properties and in particular prove
a progress result: every term is either a value or can take a reduction step.
%
By repeatedly applying the progress result, until we either reach a value or the end
user runs out of patience and kills the process, this proof freely gives us an
evaluator that is guaranteed to be correct with respect to \Velo{}'s operational
semantics.

Following existing approaches~\cite[Part 2: Properties]{plfa22.08}, we have defined
inductive families describing how terms reduce.

\ExecuteMetaData[Code/MiniExecute.idr.tex]{ReduxDef}

As can be seen above, our setting enforces call-by-value:
as described by the rule \IdrisData{ReduceFuncApp}
(\exprApp{\exprLam{b}}{t}) only reduces to
($b \, \lbrace x \leftarrow t \rbrace$)
if $t$ is already known to be a value.
%
Furthermore, our algebraic design (\Cref{sec:design:constants}) allows
us to easily enforce a left-to-right evaluation order by having a generic
family describing how primitive operations' arguments reduce.
%
As can be seen below: when considering a type-aligned list of arguments,
either the \IdrisBound{hd} takes a step and the \IdrisBound{rest} is unchanged,
or the \IdrisBound{hd} is already known to be a value and a further argument
is therefore allowed to take a step.

\ExecuteMetaData[Code/MiniExecute.idr.tex]{ReducesDef}

We differ, however, from standard approaches by genericsing our proofs of progress such that the boilerplate for computing the reflexive transitive closure when reducing terms is tidied away in a shareable module.
%
Our top-level progress definition is thus parameterised by reduction and value definitions:

\ExecuteMetaData[Code/MiniExecute.idr.tex]{ProgressDef}

\noindent
and the result of execution, which is similarly parameterised, is as follows
(where \IdrisType{RTList} is the type taking a relation and returning its
reflexive-transitive closure):

\ExecuteMetaData[Code/MiniExecute.idr.tex]{ResultDef}

The benefit of our approach is that language designers need only to provide details of what reductions are, and how to compute a single reduction, the rest comes for free.
%
Moreover, with the result of evaluation we also get the list of reduction steps made that can, optionally, be printed to show a trace of execution.

%%% Local Variables:
%%% mode: latex
%%% TeX-master: "../paper"
%%% End:


\section{Good Programming Idioms}
\label{sec:idioms}

We end our tour of \Velo{} by detailing two programming idioms that helped with the reporting of errors, and reasoning about decidable equality of inductive data structures.

\subsection{Being Positively Negative}
\label{sec:idioms:posneg}

A workhorse of dependently typed programming is using decidable
predicates that capture, and produce, positive information.
We can represent the results of a decidable function using \texttt{Dec}:

\begin{Verbatim}
data Dec a = Yes a | No (a -> Void)
\end{Verbatim}

\noindent
Notice that in the negative position (the \texttt{No} constructor) we do not return positive information.
We provide a proof of falsity.
When reasoning about our programs such proof of falsity is good.
When interacting with our programs such proof of falsity is not good.
Consider the following type signatures that specify two decidable procedures.
The first to determine if one natural number is greater than another.

\begin{Verbatim}
isGT : (x,y : Nat) -> Dec (GT x y)
\end{Verbatim}

\noindent
The other, \IdrisFunction{any}, determines if any element in the list satisfies a supplied predicate.

\begin{Verbatim}
any : (f  : (x : type) -> Dec (p x)) -> (xs : List type) -> Dec (Any p xs)
\end{Verbatim}

For \IdrisFunction{isGT} we know that if the result is negative we can safely assume that \IdrisType{GT} \IdrisImplicit{x y} is false.
With \IdrisFunction{any} we do not know for sure why the decidable procedure failed.
Was it because the list was empty?
or,
was it because all elements of \IdrisBound{xs} did not satisfy \IdrisImplicit{p}!?
Furthermore, we cannot programmatically determine the reason why an element in \IdrisBound{xs} did not satisfy \IdrisImplicit{p}?

To address these issues. we introduce \IdrisType{DecInfo} a variant of \IdrisType{Dec} that carries positive information in the negative case, as well as proofs-of-falsity.

\chcomment[id=gallais]
          {Do we want to spend precious space talking about DecInfo
            rather than going straight to the nicest solution?}
\chcomment[id=jfdm]
          {But we do not use the nicest solution...}

\begin{Verbatim}
data DecInfo e p = Yes p | No e (p -> Void)
\end{Verbatim}

With this simple change we can now report \emph{why} a decision procedure failed, as well as proof that it failed.
Our examples can now become:

\begin{Verbatim}
isGT : (x,y : Nat) -> DecInfo (LTE x y) (GT x y)
\end{Verbatim}

\noindent
and

\begin{Verbatim}
any : (f  : (x : type) -> Dec (q x) (p x))
   -> (xs : List type)
         -> DecInfo (All (\textbackslash{}x => Pair (q x) (Not (p x))) xs) (Any p xs)
\end{Verbatim}

\IdrisType{DecInfo} does, however, carry a proof of false.
A more interesting line of work would be to adopt \emph{constructive negation} to ensure that proofs of false arise from positive sources only~\cite{msfp/Atkey22}.
That is, given:

\begin{Verbatim}
Predicate : Type
Predicate = (pos ** neg ** (pos -> neg -> Void))
\end{Verbatim}

\noindent
we can recreate \IdrisType{DecInfo} as:

\begin{Verbatim}
DecInfo : Predicate -> Type
DecInfo (pos ** neg ** no) = Either neg pos
\end{Verbatim}

\noindent
Being \emph{positively negative} stems back to producing good error messages when implementing elaborators~\cite{DBLP:journals/jfp/McBrideM04}.
We are investigating the efficacy of being \emph{positively negative} and how that impacts program design in dependently typed languages.

%%% Local Variables:
%%% mode: latex
%%% TeX-master: "../paper"
%%% End:

\input{./content/efficientdeceq}

\section{Conclusion}
\label{sec:conclusion}

% we have 2 pages of refs for free
\newpage
\bibliography{paper}

\newpage
\appendix

\begin{figure}[ht]
  \centering
  \newcommand{\syntaxtypes}{
\[\begin{array}{lcl}
  \ty{t}{\Type}
  & \Coloneqq
  & \TyNat \\
  & \fpAlt
  & \TyBool \\
  & \fpAlt
  & \typeFuncIntro{}
\end{array}\]}

\newcommand{\syntaxcontexts}{
\[\begin{array}{lcl}
  \ty{\Gamma}{\Type}
  & \Coloneqq
  & \epsilon \\
  & \fpAlt
  & \Gamma,\, \ty{x}{t}
\end{array}\]}

\newcommand{\inferenceRule}{
  $\Gamma \vdash \ty{t}{a}$
}

\newcommand{\varRule}{
  $\Gamma \ni \ty{x}{a}$
}

\newcommand{\inferenceZero}{
  \[
  \infer{ }{\Gamma \vdash \ty{\exprZero}{\TyNat}}
  \]
}

\newcommand{\inferenceVar}{
  \[
  \infer{\Gamma \ni \ty{x}{a}}{\Gamma \vdash \ty{x}{a}}
  \]
}

\newcommand{\inferenceInc}{
  \[
  \infer{\Gamma \vdash \ty{n}{\TyNat}
    }{\Gamma \vdash \ty{\exprInc{n}}{\TyNat}}
  \]
}

\newcommand{\inferenceApp}{
  \[
  \infer{\Gamma \vdash \ty{f}{\TyFunc{a}{b}}
        \\ \Gamma \vdash \ty{t}{a}
    }{\Gamma \vdash \ty{\exprApp{f}{t}}{b}}
  \]
}

\newcommand{\inferenceFunc}{
  \[
  \infer{\Gamma,\, \ty{x}{a} \vdash \ty{t}{b}
    }{\Gamma \vdash \ty{\exprLam{t}}{\TyFunc{a}{b}}}
  \]
}

\newcommand{\syntaxlang}{
\begin{align*}
  \ty{e}{t}
  &
    \Coloneqq
    x
    \fpAlt
    \exprZero
    \fpAlt
    \exprIncIntro{}
    \fpAlt
    \exprAddIntro{}
  &
    \text{Expressions}
  \\
  &
    \firstAlt
    \exprTrue{}
    \fpAlt
    \exprFalse{}
    \fpAlt
    \exprAndIntro{}
  &
  \\
  &
    \firstAlt
    \exprLamIntro{}
    \fpAlt
    \exprLetIntro{}
    \fpAlt
    \exprAppIntro{}
  &
  \\
  \ty{v}{t}
  &
    \Coloneqq
    \exprZero
    \fpAlt
    \exprIncValue{}
    \fpAlt
    \exprTrue{}
    \fpAlt
    \exprFalse{}
    \fpAlt
    \exprLamValue{}
  &
    \text{Values}
\end{align*}
}

%%% Local Variables:
%%% mode: latex
%%% TeX-master: "../paper"
%%% End:

  \caption{\label{fig:velo:syntax}\Velo{} Formal Abstract Syntax}
\end{figure}

\begin{figure}[ht]
  \centering
  \input{./figure/statics}
  \caption{\label{fig:velo:statics}\Velo{} Typing Rules}
\end{figure}

\begin{figure}[ht]
  \centering
  \input{./figure/dynamics-smallstep}
  \caption{\label{fig:velo:statics}\Velo{} SmallStep Reduction Semantics}
\end{figure}

\end{document}
