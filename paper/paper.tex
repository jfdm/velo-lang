\documentclass[%
draft, % note that the orcid.pdf logo is not included with draft mode
a4paper,
UKenglish,
cleveref,
autoref,
thm-restate,
%anonymous,
pdfa
]{oasics-v2021}

\usepackage{todonotes}
\setuptodonotes{inline}

\usepackage{acronym}
\usepackage{xspace}
\usepackage[commandnameprefix=always]{changes}

\pdfoutput=1
\hideOASIcs %uncomment to remove references to OASIcs series (logo, DOI, ...), e.g. when preparing a pre-final version to be uploaded to arXiv or another public repository

%\graphicspath{{./graphics/}}%helpful if your graphic files are in another directory

\bibliographystyle{plainurl}% the mandatory bibstyle

\title{Type Theory as a Language Workbench} %TODO Please add

\author
{Jan {de Muijnck-Hughes}}
{University of Glasgow, UK}
{Jan.deMuijnck-Hughes@glasgow.ac.uk}
{https://orcid.org/0000-0003-2185-8543}
{} % TODO

\author
{Guillaume Allais}
{University of St Andrews, UK}
{gxa1@st-andrews.ac.uk} % Check
{https://orcid.org/0000-0002-4091-657X} % ORCID
{} % TODO

\author
{Edwin Brady}
{University of St Andrews, UK}
{ecb10@st-andrews.ac.uk} % Check
{} % ORCID
{} % TODO

% TODO mandatory, please use full name; only 1 author per \author macro; first two parameters are mandatory, other parameters can be empty. Please provide at least the name of the affiliation and the country. The full address is optional. Use additional curly braces to indicate the correct name splitting when the last name consists of multiple name parts.

\authorrunning{J. {de Muijnck-Hughes}, G. Allais, and E. Brady}
% TODO mandatory. First: Use abbreviated first/middle names. Second (only in severe cases): Use first author plus 'et al.'

\Copyright{Jan de Muijnck-Hughes, Guillaume Allais, and Edwin Brady}
% TODO mandatory, please use full first names. LIPIcs license is "CC-BY";  http://creativecommons.org/licenses/by/3.0/

%\begin{CCSXML}
%<ccs2012>
%   <concept>
%       <concept_id>10003752.10003790.10011740</concept_id>
%       <concept_desc>Theory of computation~Type theory</concept_desc>
%       <concept_significance>500</concept_significance>
%       </concept>
%   <concept>
%       <concept_id>10011007.10010940.10010992.10010998.10010999</concept_id>
%       <concept_desc>Software and its engineering~Software verification</concept_desc>
%       <concept_significance>500</concept_significance>
%       </concept>
%   <concept>
%       <concept_id>10011007.10011006.10011008.10011009.10011012</concept_id>
%       <concept_desc>Software and its engineering~Functional languages</concept_desc>
%       <concept_significance>500</concept_significance>
%       </concept>
%   <concept>
%       <concept_id>10011007.10011006.10011039</concept_id>
%       <concept_desc>Software and its engineering~Formal language definitions</concept_desc>
%       <concept_significance>500</concept_significance>
%       </concept>
%   <concept>
%       <concept_id>10011007.10011006.10011050.10011017</concept_id>
%       <concept_desc>Software and its engineering~Domain specific languages</concept_desc>
%       <concept_significance>500</concept_significance>
%       </concept>
%   <concept>
%       <concept_id>10002950.10003714.10003732.10003733</concept_id>
%       <concept_desc>Mathematics of computing~Lambda calculus</concept_desc>
%       <concept_significance>500</concept_significance>
%       </concept>
% </ccs2012>
%\end{CCSXML}

\ccsdesc[500]{Theory of computation~Type theory}
\ccsdesc[500]{Software and its engineering~Software verification}
\ccsdesc[500]{Software and its engineering~Functional languages}
\ccsdesc[500]{Software and its engineering~Formal language definitions}
\ccsdesc[500]{Software and its engineering~Domain specific languages}
\ccsdesc[500]{Mathematics of computing~Lambda calculus}
\ccsdesc{Software and its engineering~Compilers}
\ccsdesc{Theory of computation~Invariants}

%\ccsdesc[100]{\textcolor{red}{Replace ccsdesc macro with valid one}}
% TODO mandatory: Please choose ACM 2012 classifications from https://dl.acm.org/ccs/ccs_flat.cfm

\keywords{dependent types, language workbenches, idris2, dsl, edsl, intrinsically scoped, well typed, co-De~Bruijn}
% TODO mandatory; please add comma-separated list of keywords

\category{}
% optional, e.g. invited paper

\relatedversion{}

\supplementdetails[
subcategory={Artifact}
]{Software}{TODO}
% \supplement{}
% optional, e.g. related research data, source code, ... hosted on a repository like zenodo, figshare, GitHub, ...
% \supplementdetails[linktext={opt. text shown instead of the URL}, cite=DBLP:books/mk/GrayR93, subcategory={Description, Subcategory}, swhid={Software Heritage Identifier}]{General Classification (e.g. Software, Dataset, Model, ...)}{URL to related version}
% linktext, cite, and subcategory are optional

% \funding{(Optional) general funding statement \dots}
% optional, to capture a funding statement, which applies to all authors. Please enter author specific funding statements as fifth argument of the \author macro.

% \acknowledgements{I want to thank \dots}
% optional

%\nolinenumbers %uncomment to disable line numbering

%Editor-only macros:: begin (do not touch as author)%%%%%%%%%%%%%%%%%%%%%%%%%%%%%%%%%%
\EventEditors{John Q. Open and Joan R. Access}
\EventNoEds{2}
\EventLongTitle{42nd Conference on Very Important Topics (CVIT 2016)}
\EventShortTitle{CVIT 2016}
\EventAcronym{CVIT}
\EventYear{2016}
\EventDate{December 24--27, 2016}
\EventLocation{Little Whinging, United Kingdom}
\EventLogo{}
\SeriesVolume{42}
\ArticleNo{23}
%%%%%%%%%%%%%%%%%%%%%%%%%%%%%%%%%%%%%%%%%%%%%%%%%%%%%%

\acrodef{dsl}[DSL]{Domain Specific Language}
\acrodef{edsl}[EDSL]{Embedded Domain Specific Language}
\acrodef{stlc}[STLC]{Simply-Typed Lambda Calculus}
\acrodef{cse}[CSE]{Common Sub-Expression Elimination}
\acrodef{ir}[IR]{Intermediate Representation}
\acrodef{repl}[REPL]{Read Eval Print Loop}
\acrodef{lsp}[LSP]{Language Server Protocol}

% Acronym & Cleverref clash is a subtle way :-(
% Ideal would be to switch to glossaries, which is my main driver, I wanted to be lightweight this time around...
%
% https://tex.stackexchange.com/questions/71364/acronym-acresetall-cleveref-multiply-defined-labels
\makeatletter
\newcommand*{\org@overidelabel}{}
\let\org@overridelabel\@verridelabel
\@ifpackagelater{acronym}{2015/03/21}{% v1.41
  \renewcommand*{\@verridelabel}[1]{%
    \@bsphack
    \protected@write\@auxout{}{\string\AC@undonewlabel{#1@cref}}%
    \org@overridelabel{#1}%
    \@esphack
  }%
}{% older versions
  \renewcommand*{\@verridelabel}[1]{%
    \@bsphack
    \protected@write\@auxout{}{\string\undonewlabel{#1@cref}}%
    \org@overridelabel{#1}%
    \@esphack
  }%
}
\makeatother

\definechangesauthor[color=orange,name={Jan}]{jfdm}
\definechangesauthor[color=green, name={Guillaume}]{gallais}
\definechangesauthor[color=red, name={Edwin}]{edwin}

\newcommand{\jfdm}[1]{\chcomment[id=jfdm]{#1}}
\newcommand{\gallais}[1]{\chcomment[id=gallais]{#1}}

\newcommand{\Velo}{V{\'e}lo\xspace}
\newcommand{\Idris}{Idris~2}
\newcommand{\DeBruijn}{De~Bruijn}
% In Dutch surnames the rules for capitalisation are as follows:
% Tussenvogels ('between letters' i.e. de,van der, van, et cetera) are capitalised if they are not between a first and last name.

\fvset
{ xleftmargin=0.5\parindent
, commandchars=\\\{\}
  % , fontsize=\smaller
  % , frame=leftline
  % , framerule=0.5mm
  % , rulecolor=\color{gray}
  % , numbers = left, numbersep=2pt
}

% Options...
\DefineVerbatimEnvironment%
  {VerbatimInline}
  {Verbatim}
  {xleftmargin=0.5\parindent
  , frame=leftline
  }
\DefineVerbatimEnvironment%
  {VerbatimInlineNumbered}
  {Verbatim}
  {xleftmargin=\parindent
    ,fontsize=\smaller
    ,numbers=left,numbersep=2pt
    ,frame=lines
  }

  % Does not work for Verbatim things, or grahpics.
\NewEnviron{centertight}[1][0.5em]
{\vspace{#1}
 {\centering\BODY}
 \vspace{#1}
}
%%% Local Variables:
%%% mode: latex
%%% TeX-master: "paper"
%%% End:


\begin{document}

%% Temporary shim before we bring in Katla's idris2.sty

\newcommand{\IdrisData}[1]{\texttt{#1}}
\newcommand{\IdrisType}[1]{\texttt{#1}}


\maketitle

% 1. State the problem
% 2. Say why it is an interesting problem
% 3. Say what your solution achieves
% 4. Say what follows from your solution

\begin{abstract}
We are planning to showcase dependently typed techniques to implement \& manipulate IRs.

The key components we are planning to treat are:

\begin{itemize}
\item Efficient de Bruijn representations
\item Co de Bruijn for CSE
\item Evaluation as Progress (with computation rules in one separate function)
\item Well scoped holes
\item Linear (in number of cases) decidable equality
\item Compact constant folding
\item Positive evidence for negative statements
\item Whole pipeline (parser, elaborator, REPL, evaluator, compiler passes)
\end{itemize}

Our ongoing work is available at:

\url{https://github.com/jfdm/velo-lang}
\end{abstract}

\todo{Run spellcheck}
\todo{Use Katla to typeset Idris 2 code}

\section{Introduction}
\label{sec:introduction}

\emph{Language Workbenches}, such as Spoofax~\cite{DBLP:journals/software/WachsmuthKV14}, offers language designers an expressive environment in which to design, implement, and deploy their \Acp{dsl}~\cite{hudak1996building}.
Another interesting environment to develop \acp{dsl} is that of dependently-typed languages such as Idris~\cite{DBLP:conf/ecoop/Brady21}, Agda~\cite{DBLP:conf/afp/Norell08}, and Coq~\cite{}.
Such languages provides us with an expressive environment in which to reason not only about our software programs but also about the programming languages in which we write our programs.
We have seen the use of dependent types to provide the mechanised reasoning of core language calculi~\cite{10.1145/3093333.3009866,DBLP:conf/cpp/RouvoetPKV20,DBLP:conf/mpc/ChapmanKNW19}.
These applications of mechanised software verification are wasted if we also do not seek to run these programs as well!
If it type checks, we might as well ship it too!

Uses of dependent-typed to realise \acp{dsl} raises an important question:
\emph{Are dependently-typed languages language workbenches?}
Criteria has been developed to detail what a language workbench is~\cite{DBLP:conf/sle/ErdwegSVBBCGHKLKMPPSSSVVVWW13}.
Principally speaking a language workbench is a tool supports:
description of a language's \emph{notation}---how we present a language's concrete syntax to users;
implementation of a language's \emph{semantics}---how we realise how a language operates;
and
user interaction through an \emph{editor}.
Outside of this core criteria other language work benches support language validation, testing, and composition.
It is easily demonstrated from extant work that dependently-typed languages support the production of verified language implementations, but what about the other required and optional criteria?


\Velo{} is a minimal functional language that we have realised in Idris to showcase dependently typed techniques to implement and manipulate \acp{ir}.
This paper introduces \Velo{} but that also though the auspices of \Velo{} itself that dependently-typed languages make a fine language workbench according to the core criteria for a language workbench, and that with future engineering we can satisfy more of the optional criteria.

% C-c C-a ?todo


%%% Local Variables:
%%% mode: latex
%%% TeX-master: "paper"
%%% End:



\section{Introducing Velo}
\label{sec:velo}

\chcomment{Whole pipeline (parser, elaborator, REPL, evaluator, compiler passes)}

\jfdm{Do we need to provide the formal description of us the STLC \& our extensions sufficiently \emph{well known}?}

\Velo{} is the \ac{stlc} extended with booleans and natural numbers with their respective conjunction and addition operations as primitives, together with well-scoped-typed holes to support simple interactive language design.
Such a featherweight language helps us to better showcase how we can use dependently-typed languages as language workbenches.
\jfdm{Do we need to evidence (through citation) benefits of featherweight languages?}

\begin{verbatim}
   let a = zero
in let b = (inc a)
in let c = (add a b)

in let d = true
in let e = false
in let f = (and d e)

in let foo = (fun x : nat => x)
in (foo ?hole)

\end{verbatim}

We have presented \Velo{} as a \emph{complete} language with a standard pipeline.
A \ac{dsl} captures the language's concrete syntax, and a parser turns \ac{dsl} instances into raw unchecked terms.
Type checking elaborates these raw terms into a set of well-typed well-scoped intermediate representations: \IdrisType{Holey} supports well-scoped typed holes; and \IdrisType{Terms} our core representation that captures our language's abstract syntax.
From the core representation we provide well-scoped \ac{cse} using co-DeBruijn inexing, and we provide a verifiable evaluator to reduce terms to values.

\jfdm{Would it be good to show an example high-level trace of the output?}

\jfdm{Perhaps we can use this section as an outline and forward reference the latter sections?}

%%% Local Variables:
%%% mode: latex
%%% TeX-master: "paper"
%%% End:


\section{Language Design}
\label{sec:design}

\subsection{Efficient \DeBruijn{} Representation}
\label{sec:design:deBruijn}

A common strategy for implementing well-scoped terms is to use typed
\emph{\DeBruijn{}} indices~\cite{MANUAL:journals/math/debruijn72},
which are easily realised as an inductive family~\cite{DBLP:journals/fac/Dybjer94}
indicating where in the type-level context the variable is bound.

Concretely, we index the \IdrisType{Elem} family by a context
(once again represented as a \IdrisType{SnocList} of kinds) and
the kind of the variable it represents.

\begin{centertight}
\begin{minipage}{0.10\textwidth}
\varRule
\end{minipage}\hfill
\begin{minipage}{0.80\textwidth}
\ExecuteMetaData[Code/MiniVelo.idr.tex]{ElemDecl}
\end{minipage}
\end{centertight}

We then match each context membership inference rule to a constructor.
%
The \IdrisData{Here} constructor indicates that the variable of interest is
the most local one in scope (note the non-linear occurence of (\ty{x}{a}) on
the left hand side, and correspondingly of \IdrisBound{ty} on the right).

\begin{centertight}
\begin{minipage}{0.35\textwidth}
  \varZero
\end{minipage}\hfill
\begin{minipage}{0.55\textwidth}
\ExecuteMetaData[Code/MiniVelo.idr.tex]{varZero}
\end{minipage}
\end{centertight}

The \IdrisData{There} constructor skips past the most local variable to look for the variable of interest deeper in the context.

\begin{centertight}
\begin{minipage}{0.35\textwidth}
  \varSuc
\end{minipage}\hfill
\begin{minipage}{0.55\textwidth}
\ExecuteMetaData[Code/MiniVelo.idr.tex]{varSuc}
\end{minipage}
\end{centertight}


Whilst a valid definition, this approach unfortunately does not scale to
large contexts:
%
every \IdrisType{Elem} proof is linear in the size of the \DeBruijn{}
index that it represents.
%
To improve the runtime efficiency of the representation we instead opt to
model \DeBruijn{} indices as natural numbers, which \Idris{} compiles to
GMP-style unbounded integers.
%
Further, we need to additionally define an \IdrisType{AtIndex} family to ensure that
all of the natural numbers we use correspond to valid indices.
%
We pointedly reuse the \IdrisType{Elem} names because these \IdrisData{Here}
and \IdrisData{There} constructors play exactly the same role.

\ExecuteMetaData[Code/MiniDeBruijn.idr.tex]{AtIndexDef}

\noindent
We then define a variable as the pairing of a natural number and an \emph{erased}
(as indicated by the \IdrisKeyword{0} annotation on the binding site for \IdrisBound{prf})
proof that the given natural number is indeed a valid \DeBruijn{} index.

\ExecuteMetaData[Code/MiniDeBruijn.idr.tex]{IsVarDef}

Thanks to Quantitative Type Theory~\cite{DBLP:conf/birthday/McBride16,DBLP:conf/lics/Atkey18}
as implemented in \Idris{}, the compiler knows that it can safely erase these
runtime-irrelevant proofs.
%
we now have the best of both worlds: a well-scoped realisation of \DeBruijn{} indices
that is compiled efficiently.

\todo{Talk about smart constructors \& views?}
\jfdm{We should do this talking in the extended version...}

Just like the naïve definition of \DeBruijn{} indexing is not the
best suited for a practical implementation,
the inductive family \IdrisType{Term} described in Section~\ref{sec:design}
is not the most convenient to use.
%
We will now see one of its limitations and how we remedied it in
\Velo{}.

%%% Local Variables:
%%% mode: latex
%%% TeX-master: "../paper"
%%% End:


\subsection{Compact Constant Folding}
\label{sec:design:constants}

Software Foundations's \emph{Programming Language Foundations} opens with a constant-folding transformation exercise~\cite[Chapter~1]{Pierce:SF2}.
%
Starting from a small language of expressions containing natural numbers, variables, addition, subtraction, and multiplication we are to deploy the semiring properties to simplify expressions.
%
The definition of the simplifying traversal contains much duplicated code due to the way the source language is structured:
%
all the binary operations are separate constructors, whose subterms need to be structurally simplified before we can decide whether a rule applies.
%
The correction proof has just as much duplication because it needs to follow the structure of the call graph of the function it wants to see reduce.
%
The only saving grace here is that Coq's tactics language lets users write scripts that apply to many similar goals thus avoiding duplication in the source file.

In \Velo{}, we structured our core language's representation in such a way that this duplication is never needed.
%
All builtin operators (from primitive operations on builtin types to function application itself) are represented using a single \IdrisData{Call} constructor which takes an operation and a type-indexed list of subterms.

\begin{Verbatim}
data Term : (metas : List Meta) -> (ctxt : List Ty) -> Ty -> Type where
  -- ...
  Call : \{tys : _\} -> (operator : Prim                  tys  ty)
                   -> (operans  : All (Term metas ctxt) tys)
                               -> Term      metas ctxt       ty
\end{Verbatim}

The primitives include:
%
\IdrisData{Zero}---which takes no argument and returns a term of type \IdrisData{TyNat};
%
\IdrisData{And}---which takes two arguments of type \IdrisData{TyBool} and return a term
of type \IdrisData{TyBool};
%
and
%
\IdrisData{App}---which takes a function and an argument that corresponds to the type of the function's domain and returns a term that is the type of the function's co-domain.

\begin{Verbatim}
data Prim : (args : List Ty) -> (type : Ty) -> Type
  where
    Zero : Prim []                    TyNat
    And  : Prim [TyBool, TyBool]      TyBool
    App  : Prim [TyFunc dom cod, dom] cod
    -- ...
\end{Verbatim}

Using \IdrisType{Prim} structural operations can now be implemented by handling recursive calls on the subterms of \IdrisData{Call} nodes uniformly before dispatching on the operator to see whether additional simplifications can be deployed.

Similarly, all of the duplication in the correction proofs is factored out in a single place where the induction hypotheses can be invoked.

%%% Local Variables:
%%% mode: latex
%%% TeX-master: "../paper"
%%% End:



\subsection{Well-Typed Holes}
\label{sec:design:holes}

\chcomment{Well-Typed Holes}

\section{Optimisation \& Execution}
\label{sec:compiler}

\chcomment{Co de Bruijn for CSE}
\chcomment{Evaluation as Progress (with computation rules in one separate function)}

\section{Good Programming Idioms}
\label{sec:idioms}

We end our tour of \Velo{} by detailing two programming idioms that helped with the reporting of errors, and reasoning about decidable equality of inductive data structures.

\subsection{Being Positively Negative}
\label{sec:idioms:posneg}

A workhorse of dependently typed programming is using decidable
predicates that capture, and produce, positive information.
We can represent the results of a decidable function using \texttt{Dec}:

\begin{Verbatim}
data Dec a = Yes a | No (a -> Void)
\end{Verbatim}

\noindent
Notice that in the negative position (the \texttt{No} constructor) we do not return positive information.
We provide a proof of falsity.
When reasoning about our programs such proof of falsity is good.
When interacting with our programs such proof of falsity is not good.
Consider the following type signatures that specify two decidable procedures.
The first to determine if one natural number is greater than another.

\begin{Verbatim}
isGT : (x,y : Nat) -> Dec (GT x y)
\end{Verbatim}

\noindent
The other, \IdrisFunction{any}, determines if any element in the list satisfies a supplied predicate.

\begin{Verbatim}
any : (f  : (x : type) -> Dec (p x)) -> (xs : List type) -> Dec (Any p xs)
\end{Verbatim}

For \IdrisFunction{isGT} we know that if the result is negative we can safely assume that \IdrisType{GT} \IdrisImplicit{x y} is false.
With \IdrisFunction{any} we do not know for sure why the decidable procedure failed.
Was it because the list was empty?
or,
was it because all elements of \IdrisBound{xs} did not satisfy \IdrisImplicit{p}!?
Furthermore, we cannot programmatically determine the reason why an element in \IdrisBound{xs} did not satisfy \IdrisImplicit{p}?

To address these issues. we introduce \IdrisType{DecInfo} a variant of \IdrisType{Dec} that carries positive information in the negative case, as well as proofs-of-falsity.

\chcomment[id=gallais]
          {Do we want to spend precious space talking about DecInfo
            rather than going straight to the nicest solution?}
\chcomment[id=jfdm]
          {But we do not use the nicest solution...}

\begin{Verbatim}
data DecInfo e p = Yes p | No e (p -> Void)
\end{Verbatim}

With this simple change we can now report \emph{why} a decision procedure failed, as well as proof that it failed.
Our examples can now become:

\begin{Verbatim}
isGT : (x,y : Nat) -> DecInfo (LTE x y) (GT x y)
\end{Verbatim}

\noindent
and

\begin{Verbatim}
any : (f  : (x : type) -> Dec (q x) (p x))
   -> (xs : List type)
         -> DecInfo (All (\textbackslash{}x => Pair (q x) (Not (p x))) xs) (Any p xs)
\end{Verbatim}

\IdrisType{DecInfo} does, however, carry a proof of false.
A more interesting line of work would be to adopt \emph{constructive negation} to ensure that proofs of false arise from positive sources only~\cite{msfp/Atkey22}.
That is, given:

\begin{Verbatim}
Predicate : Type
Predicate = (pos ** neg ** (pos -> neg -> Void))
\end{Verbatim}

\noindent
we can recreate \IdrisType{DecInfo} as:

\begin{Verbatim}
DecInfo : Predicate -> Type
DecInfo (pos ** neg ** no) = Either neg pos
\end{Verbatim}

\noindent
Being \emph{positively negative} stems back to producing good error messages when implementing elaborators~\cite{DBLP:journals/jfp/McBrideM04}.
We are investigating the efficacy of being \emph{positively negative} and how that impacts program design in dependently-typed languages.

%%% Local Variables:
%%% mode: latex
%%% TeX-master: "../paper"
%%% End:


\subsection{Efficient Decidable Equality}
\label{sec:idioms:decEq}
\chcomment{Linear (in number of cases) decidable equality}

When users don't have access to a meta-program deriving proofs that propositional
equality is decidable~\cite{DBLP:conf/icfp/ChristiansenB16},
the most common strategy is to use nested pattern matching and produce
a number of clauses quadratic in the number of constructors for the type at hand.


\chcomment[id=gallais]{I don't understand this claim:
  \begin{quote}
      Although we can already reduce the number of contraditions using
      symmetry breaking (\texttt{negEqSym}), the number of cases is still many.
   \end{quote}
   }

We can reduce the complexity of \IdrisType{DecEq} instance creation from quadratic
to linear in the number of constructors.
%
We first define a \IdrisType{Diag} relation stating that two terms have the same
top-level constructor.
%
We can then define a \IdrisType{diag} function that, from two terms, either returns
a proof that they satisfy the \IdrisType{Diag} relation or return \IdrisData{Nothing}.
%
We can easily proof that \IdrisType{diag} cannot possibly return \IdrisData{Nothing}
if its input actually are equal.
%
We can finally use this auxiliary function to implement \IdrisFunction{decEq}
by only needing to consider cases where the two input terms share the same
top-level constructor and have a generic catch-all case handling all top-level
mismatches thanks to \IdrisFunction{diagNot}.


\chcomment[id=jfdm]{The example is probably not needed, but useful to have...}
\chcomment[id=gallais]{I think an example with recursive subterms would be more helpful}

\begin{verbatim}
data Thing = Apples | Oranges

data Diag : (x,y : Thing) -> Type where
  A : Diag Apples Apples
  O : Diag Oranges Oranges

diag : (x,y : Thing) -> Maybe (Diag x y)
diag Apples Apples = pure A
diag Oranges Oranges = pure O
diag _ _ = Nothing

diagNot : (x : Thing) -> Not (diag x x === Nothing)
diagNot Apples Refl impossible
diagNot Oranges Refl impossible

diagEq : Diag x y -> Dec (x === y)
diagEq A = Yes Refl
diagEq O = Yes Refl

decEq : (x,y : Thing) -> Dec ( x === y)
decEq x y with (diag x y) proof sim
  decEq x y | Nothing = No (\Refl => diagNot _ sim)
  decEq x y | (Just prf) = diagEq prf
\end{verbatim}

\begin{verbatim}
data Bin = Leaf | Node Bin Bin

data Diag : (s, t : Bin) -> Type where
  Leaf2 : Diag Leaf Leaf
  Node2 : (s, t, u, v : Bin) -> Diag (Node s t) (Node u v)

diag : (s, t : Bin) -> Maybe (Diag s t)
diag Leaf Leaf = Just Leaf2
diag (Node s t) (Node u v) = Just (Node2 s t u v)
diag _ _ = Nothing

diagNot : (t : Bin) -> Not (diag t t === Nothing)
diagNot Leaf = absurd
diagNot (Node _ _) = absurd

decEq : (s, t : Bin) -> Dec (s === t)
decEq s@_ t@_ with (diag s t) proof eq
  _ | Just Leaf2 = Yes Refl
  _ | Just (Node2 a b u v) with (decEq a u) | (decEq b v)
    _ | Yes eq1 | Yes eq2 = Yes (cong2 Node eq1 eq2)
    _ | No neq1 | _ = No (\case Refl => neq1 Refl)
    _ | _ | No neq2 = No (\case Refl => neq2 Refl)
  _ | Nothing = No (\ Refl => diagNot _ eq)
\end{verbatim}


\section{Conclusion}
\label{sec:conclusion}


\bibliography{paper}

\end{document}
