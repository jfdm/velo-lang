\subsection{Efficient \DeBruijn{} Representation}
\label{sec:design:deBruijn}

A common strategy for implementing well-scoped terms is to use typed
\emph{\DeBruijn{}} indices~\cite{MANUAL:journals/math/debruijn72},
which are easily realised as an inductive family~\cite{DBLP:journals/fac/Dybjer94}
indicating where in the type-level context the variable is bound.

Concretely, we index the \IdrisType{Elem} family by a context
(once again represented as a \IdrisType{SnocList} of kinds) and
the kind of the variable it represents.

\begin{center}
\begin{minipage}{0.10\textwidth}
\varRule
\end{minipage}\hfill
\begin{minipage}{0.80\textwidth}
\ExecuteMetaData[Code/MiniVelo.idr.tex]{ElemDecl}
\end{minipage}
\end{center}

We then match each context membership inference rule to a constructor.
%
The \IdrisData{Here} constructor indicates that the variable of interest is
the most local one in scope (note the non-linear occurence of (\ty{x}{a}) on
the left hand side, and correspondingly of \IdrisBound{ty} on the right).

\begin{center}
\begin{minipage}{0.35\textwidth}
  \varZero
\end{minipage}\hfill
\begin{minipage}{0.55\textwidth}
\ExecuteMetaData[Code/MiniVelo.idr.tex]{varZero}
\end{minipage}
\end{center}

The \IdrisData{There} constructor skips past the most local variable to look for the variable of interest deeper in the context.

\begin{center}
\begin{minipage}{0.35\textwidth}
  \varSuc
\end{minipage}\hfill
\begin{minipage}{0.55\textwidth}
\ExecuteMetaData[Code/MiniVelo.idr.tex]{varSuc}
\end{minipage}
\end{center}


Whilst a valid definition, this approach unfortunately does not scale to
large contexts:
%
every \IdrisType{Elem} proof is linear in the size of the \DeBruijn{}
index that it represents.
%
To improve the runtime efficiency of the representation we instead opt to
model \DeBruijn{} indices as natural numbers, which \Idris{} compiles to
GMP-style unbounded integers.
%
Further, we need to additionally define an \IdrisType{AtIndex} family to ensure that
all of the natural numbers we use correspond to valid indices.
%
We pointedly reuse the \IdrisType{Elem} names because these \IdrisData{Here}
and \IdrisData{There} constructors play exactly the same role.

\ExecuteMetaData[Code/MiniDeBruijn.idr.tex]{AtIndexDef}

\noindent
We then define a variable as the pairing of a natural number and an \emph{erased}
proof that the given natural number is indeed a valid \DeBruijn{} index.

\ExecuteMetaData[Code/MiniDeBruijn.idr.tex]{IsVarDef}

Thanks to Quantitative Type Theory~\cite{DBLP:conf/birthday/McBride16,DBLP:conf/lics/Atkey18}
as implemented in \Idris{},
we now have the best of both worlds: a well-scoped realisation of \DeBruijn{} indices
that is guaranteed to be compiled efficiently.

\todo{Talk about smart constructors \& views?}
\jfdm{We should do this talking in the extended version...}

Just like the naïve definition of \DeBruijn{} indexing is not the
best suited for a practical implementation,
the inductive family \IdrisType{Term} described in Section~\ref{sec:design}
is not the most convenient to use.
%
We will now see one of its limitations and how we remedied it in
\Velo{}.

%%% Local Variables:
%%% mode: latex
%%% TeX-master: "../paper"
%%% End:
