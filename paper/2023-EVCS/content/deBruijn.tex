\subsection{Efficient \DeBruijn{} Representation}
\label{sec:design:deBruijn}

A common strategy for implementing well-scoped terms is to use typed
\emph{\DeBruijn{}} indices~\cite{MANUAL:journals/math/debruijn72}, and these indices can be easily realised as an inductive family~\cite{DBLP:journals/fac/Dybjer94}
indicating where in the type-level context the variable is bound.

Concretely, we index the \IdrisType{Elem} family by the kind of the variable it represents and a \IdrisType{SnocList} of kinds as the context to reflect the fact that,
in inference rules, the most local end of the context is always on the right hand side.

\begin{minipage}{0.10\textwidth}
\varRule
\end{minipage}\hfill
\begin{minipage}{0.80\textwidth}
\begin{Verbatim}
data Elem : (ctxt : SnocList a) -> (type : a) -> Type where
\end{Verbatim}
\end{minipage}

The \IdrisData{Here} constructor indicates that the variable of interest is
the most local one in scope.
%

\begin{minipage}{0.35\textwidth}
  \varZero
\end{minipage}\hfill
\begin{minipage}{0.55\textwidth}
\begin{Verbatim}
  Here : ----------------------
          Elem ty (ctxt :< ty)
\end{Verbatim}
\end{minipage}

The \IdrisData{There} constructor skips past the most local one to look for
the variable of interest in the rest of the context.

\begin{minipage}{0.35\textwidth}
  \varSuc
\end{minipage}\hfill
\begin{minipage}{0.55\textwidth}
\begin{Verbatim}
  There :  Elem ty ctxt ->
          ---------------------
           Elem ty (ctxt :< _)
\end{Verbatim}
\end{minipage}


Whilst a valid definition, this approach unfortunately does not scale to
large contexts:
%
every \IdrisType{Elem} proof is linear in the size of the \DeBruijn{}
index that it represents.
%
To improve the runtime efficiency of the representation we instead opt to
model \DeBruijn{} indices as natural numbers, which \Idris{} will compile to
GMP-style unbounded integers.
%
Further, we need to additionally define an \IdrisType{AtIndex} family to ensure that
all of the natural numbers we use correspond to valid indices.
%
We pointedly reuse the \IdrisType{Elem} names because these \IdrisData{Here}
and \IdrisData{There} play exactly the same role.

\begin{Verbatim}
data AtIndex : (ty : kind) -> (ctxt : SnocList kind) ->
               (idx : Nat) -> Type where
  Here : AtIndex ty (ctxt :< ty) Z
  There : (later : AtIndex ty ctxt idx) -> AtIndex ty (ctxt :< _) (S idx)
\end{Verbatim}

\noindent
We then define a variable as the pairing of a natural number and an \emph{erased}
proof that the given natural numbers is indeed a valid \DeBruijn{} index.

\begin{Verbatim}
data IsVar : (ctxt : SnocList kind) -> (ty : kind) -> Type where
    V : (idx : Nat) -> (0 prf : AtIndex ty ctxt idx) -> IsVar ctxt ty
\end{Verbatim}

Thanks to Quantitative Type Theory~\cite{DBLP:conf/birthday/McBride16,DBLP:conf/lics/Atkey18}
as implemented in \Idris{},
we now have the best of both worlds: a well-scoped realisation of \DeBruijn{} indices
that is guaranteed to be compiled efficiently.

\todo{Talk about smart constructors \& views?}
\jfdm{We should do this talking in the extended version...}

%%% Local Variables:
%%% mode: latex
%%% TeX-master: "../paper"
%%% End:
