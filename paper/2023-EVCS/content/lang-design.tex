We begin our discussion by detailing the key design rationale on
realising the static semantics of \Velo{} within \Idris{}.

We have opted to give \Velo{} an external concrete syntax (a \ac{dsl})
in which users can write their programs.
%
With dependently typed languages we can also capture
the abstract syntax and its static semantics as an intrinsically
scoped and typed \ac{edsl}
directly within the host language~\cite{Augustsson1999edt}.
%
That is to say that the data structure is designed in such a way that
we can only represent well scoped and well typed terms and, correspondingly,
that our scope- and type- checking passes are guaranteed to have rejected
invalid user inputs.
%
To keep the exposition concise, we focus on a core subset of the
language. The interested reader can find the whole definition
in the accompanying material\todo{Add DOI}.

\textbf{Types} are usually introduced using their context free grammar.
%
We present it here on the left-hand side, it gives users the choice between
two base types (\TyNat, and \TyBool) and a type former for function types
(\TyFunc{\cdot}{\cdot}).
%
On the right hand side, we give their internal representation as an inductive
type in \Idris{}.%
\footnote{
Throughout this article, the \Idris{} code snippets are
automatically rendered using a semantic highlighter.
%
Keywords are typeset in \IdrisKeyword{bold},
types in \IdrisType{blue},
data constructors in \IdrisData{red},
function definitions in \IdrisFunction{green},
bound variables in \IdrisBound{purple},
and comments in \IdrisComment{grey}.
}

\begin{center}
\begin{minipage}{0.45\textwidth}
\syntaxtypes
\end{minipage}\hfill
\begin{minipage}{0.45\textwidth}
\ExecuteMetaData[Code/MiniVelo.idr.tex]{TyDef}
\end{minipage}
\end{center}

\textbf{Contexts} can be similarly given by a context free grammar:
a context is either empty ($\epsilon$), or an existing context ($\Gamma$)
extended on the right with a new type assignment (\ty{x}{t}) using a comma.
%
In \Idris{}, we will adopt a nameless representation and so we represent
these contexts by using a \IdrisData{SnocList} of types
(i.e.\ lists that grow on the right).
%
Note that the \Idris{} compiler automatically supports sugar for lists and
snoc lists: \IdrisData{[1,2,3]} represents a list counting up from
\IdrisData{1} to \IdrisData{3} while \IdrisData{[<1,2,3]} is its snoc list
pendant counting down.
%
In particular \IdrisData{[<]} denotes the empty snoc list also known as \IdrisData{Lin}.

\begin{center}
\begin{minipage}{0.35\textwidth}
\syntaxcontexts
\end{minipage}\hfill
\begin{minipage}{0.55\textwidth}
\ExecuteMetaData[Code/MiniVelo.idr.tex]{SnocListDef}
\end{minipage}
\end{center}

\todo{play with spaces, reread descriptions}

\textbf{Typing Judgements} are given by relations, and encoded in
\Idris{} using inductive families, a generalisation of inductive
types~\cite{DBLP:journals/fac/Dybjer94}.
%
Each rule will become a constructor for the family, and so every
proof \inferenceRule{} will correspond to a term $t$ of type
(\texttt{Term} $\Gamma$ $a$).
%
On the left hand side we present two judgements: context membership
and a typing judgement, and on the right we have the corresponding
inductive family declarations.

\begin{center}
\begin{minipage}{0.10\textwidth}
\varRule
\inferenceRule
\end{minipage}\hfill
\begin{minipage}{0.80\textwidth}
\ExecuteMetaData[Code/MiniVelo.idr.tex]{ElemTermDecl}
\end{minipage}
\end{center}

We leave the definition of \IdrisType{Elem} to the next section,
focusing instead on \IdrisType{Term}.
%
The most basic of typing rules are axioms, they have no premise
and are mapped to constructors with no argument.
We use \Idris{} comments (\IdrisComment{\KatlaDash{}\KatlaDash})
to format our constructor's type in such
a way that they resemble the corresponding inference rule.
%
Here we show the rule stating that $0$ is a natural number and
its translation as the \IdrisData{Zero} constructor.

\begin{center}
\begin{minipage}{0.45\textwidth}
\inferenceZero
\end{minipage}\hfill
\begin{minipage}{0.45\textwidth}
\ExecuteMetaData[Code/MiniVelo.idr.tex]{inferenceZero}
\end{minipage}
\end{center}

Then come typing rules with a single premise which is not a subderivation
of the relation itself.
They are mapped to constructors with a single argument.
%
Here we show the typing rule for variables: given a proof that we have a
variable of type $a$ somewhere in the context, we can build a term of type
$a$ in said context.

\begin{center}
\begin{minipage}{0.45\textwidth}
\inferenceVar
\end{minipage}\hfill
\begin{minipage}{0.45\textwidth}
\ExecuteMetaData[Code/MiniVelo.idr.tex]{inferenceVar}
\end{minipage}
\end{center}

Next, we have typing rules with a single premise which is a subderivation.
They are mapped to constructors with a single argument of the inductive family
representing the subderivation.
%
Here we show the typing rule for successor: provided that we are given
a natural number in a given context, its successor is also a natural
number in the same context.

\begin{center}
\begin{minipage}{0.45\textwidth}
\inferenceInc
\end{minipage}\hfill
\begin{minipage}{0.45\textwidth}
\ExecuteMetaData[Code/MiniVelo.idr.tex]{inferenceInc}
\end{minipage}
\end{center}

Similarly, rules with two premises are translated to constructors
with two arguments, one for each subderivation.
%
Here we present the typing rule for application nodes: provided that
the function has a function type, and the argument has a type matching
the function's domain, the application has a type corresponding to the
function's codomain.
Note that the context $\Gamma$ is the same across the whole rule and
so the same \IdrisBound{gamma} is used everywhere.


\begin{center}
\begin{minipage}{0.35\textwidth}
\inferenceApp
\end{minipage}\hfill
\begin{minipage}{0.55\textwidth}
\ExecuteMetaData[Code/MiniVelo.idr.tex]{inferenceApp}
\end{minipage}
\end{center}

Finally, we have a rule where the premise's context has been extended:
a function of type (\TyFunc{a}{b}) is built by providing a term
of type $b$ defined in a context extended with a new variable of type $a$.

\begin{center}
\begin{minipage}{0.35\textwidth}
\inferenceFunc
\end{minipage}\hfill
\begin{minipage}{0.55\textwidth}
\ExecuteMetaData[Code/MiniVelo.idr.tex]{inferenceFunc}
\end{minipage}
\end{center}

Using this intrinsically typed representation, we can readily represent
entire typing derivations.
%
The following example\footnote{\IdrisData{Here} will be defined in
Section~\ref{sec:design:deBruijn} as a constructor for the \IdrisType{Elem} family.}
presents the internal representation
\IdrisFunction{Plus2} of the derivation proving that
$\exprLam{\exprInc{\exprInc{x}}}$ can be
assigned the type ($\TyFunc{\TyNat}{\TyNat}$).

\begin{center}
\begin{minipage}{0.4\textwidth}
\infer
  {\infer{\vdots}{\epsilon\,, \ty{x}{\TyNat} \vdash \ty{\exprInc{\exprInc{x}}}{\TyNat}}}
  {\epsilon \vdash \ty{\exprLam{\exprInc{\exprInc{x}}}}{\TyFunc{\TyNat}{\TyNat}}}
\end{minipage}\hfill
\begin{minipage}{0.5\textwidth}
\ExecuteMetaData[Code/MiniVelo.idr.tex]{Plus2Def}
\end{minipage}
\end{center}

By using \IdrisType{Term} as an \ac{ir} in our compiler
we have made entire classes of invalid programs unrepresentable:
it is impossible to form an ill scoped or ill typed term.
%
Indeed, trying to write an ill scoped or an ill typed program leads to a
static error as demonstrated by the following \IdrisKeyword{failing} blocks.%
\footnote{\Idris{} only accepts failing blocks if checking
their content yields an error matching the given string.}
%
In this first example we try to refer to a variable in an empty context.
\Idris{} correctly complains that this is not possible.

\begin{center}
\ExecuteMetaData[Code/MiniVelo.idr.tex]{IllScoped}
\end{center}

In this second example we try to type the identity function as a function
from \TyNat to \TyBool. This is statically rejected as nonsensical:
\IdrisData{TyNat} and \IdrisData{TyBool} are distinct constructors!

\begin{center}
\ExecuteMetaData[Code/MiniVelo.idr.tex]{IllTyped}
\end{center}


\noindent
Using such intrinsically typed \acp{edsl} we can statically enforce that
our elaborators do check that the raw terms obtained by parsing user input
are well scoped and well typed.
%
Writing our compiler passes (model-to-model transformations) and
evaluation engine (model-to-host transformation) using these
invariant-rich \acp{ir} additionally ensures that each step respects
the language's static semantics.
%
In fact we will describe in \Cref{sec:semantics} how we can use our \acp{edsl}
to both verify our static semantics whilst describing our dynamic semantics.

For languages equipped with more advanced type systems, that cannot be as easily
enforced statically, we can retain some of these guarantees by using a well
scoped core language rather than a well typed one.
%
This is the approach used in \Idris{} and it has already helped eliminate an
entire class of bugs arising when attempting to solve a metavariable with a
term that was defined in a different context~\cite{DBLP:conf/ecoop/Brady21}.
