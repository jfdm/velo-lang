\emph{Language Workbenches}, such as
Spoofax~\cite{DBLP:journals/software/WachsmuthKV14},
offer language designers an expressive environment in which to design,
implement, and deploy their \Acp{dsl}~\cite{hudak1996building}.
%
Principally speaking a language workbench~\cite{DBLP:conf/sle/ErdwegSVBBCGHKLKMPPSSSVVVWW13}
is a tool that supports:
description of a language's \emph{notation}---how we present a language's concrete syntax to users;
implementation of a language's \emph{semantics}---how we realise the language's behaviour;
and user interaction through an \emph{editor}.
%
Outside of these core criteria, various language workbenches
support language validation, testing, and composition.


Concurrently, the mechanised meta-theory research
programme~\cite{DBLP:conf/tphol/AydemirBFFPSVWWZ05,DBLP:journals/jfp/AbelAHPMSS19}
has seen a wealth of tools and techniques being developed
in the programming languages theory community.
%
In particular, dependently typed languages such as
\Idris{}~\cite{DBLP:conf/ecoop/Brady21},
Agda~\cite{DBLP:conf/afp/Norell08},
and Coq~\cite{the_coq_development_team_2022_5846982}
have been widely used to formalise \acp{dsl}, and study their
meta-theoretical properties.
%
Dependent types allow types to depend on values and provide an expressive environment in which to reason about, and write, our software programs.
%
Efforts using dependently-typed languages range from
studying specific core calculi~\cite{10.1145/3093333.3009866,DBLP:conf/cpp/RouvoetPKV20,DBLP:conf/mpc/ChapmanKNW19}
to building generic reasoning frameworks~\cite{DBLP:conf/cpp/StarkSK19,DBLP:journals/jfp/AllaisACMM21}.
%
These mechanised software verification projects, however, typically stop short
of building the frontend that would let users run these verified
language implementations.
If our verified language implementations type check, we might as well ship them too!
%
By becoming its own implementation language, \Idris{} has successfully
demonstrated that this is not an inescapable fate~\cite{DBLP:conf/ecoop/Brady21}.
%
But can we now claim that dependently typed languages qualify as
language workbenches?

\Velo{}\footnote{\url{https://github.com/jfdm/velo-lang}} is a minimal functional language that we have realised in \Idris{}
to showcase dependently typed techniques to implement and manipulate \acp{ir}.
%
This paper introduces \Velo{} but, most of all, seeks to show that
dependently typed languages make fine language workbenches.
%
We address both the core criteria and some optional extensions
highlighted by the language workbench challenge~\cite{DBLP:conf/sle/ErdwegSVBBCGHKLKMPPSSSVVVWW13} for what constitutes a language workbench.
%
%
Although not all of the optional criteria
%for a language workbench
are met by dependently typed languages, we are convinced that
with some additional engineering (taking advantage of existing work,
for example Quickchick~\cite{DBLP:journals/pacmpl/LampropoulosPP18})
more optional criteria can be satisfied.

Another key tenet in language workbenches, such as Spoofax,
is the \emph{ease} with which languages can be created.
%
To that same degree, we have developed a series
of reusable modules
%, where possible,
that captures
functionality common to many languages.
%
Thereby reducing the \emph{boilerplate} required when creating \acp{edsl} in \Idris{}.

Although we have made effort to make dependently-typed programming accessible in our presentation, more introductory material is available for the interested reader~\cite{plfa22.08,brady17:_type_driven_devel_idris}.

\todo{Add link to DOI-backed artifact}

%%% Local Variables:
%%% mode: latex
%%% TeX-master: "../paper"
%%% End:
