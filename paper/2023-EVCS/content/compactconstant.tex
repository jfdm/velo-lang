\subsection{Compact Constant Folding}
\label{sec:design:constants}

Software Foundations' \emph{Programming Language Foundations}
opens with a constant-folding transformation exercise~\cite[Chapter~1]{Pierce:SF2}.
%
Starting from a small language of expressions (containing natural numbers, variables, addition, subtraction, and multiplication) we are to deploy the semiring properties to simplify expressions.
%
The definition of the simplifying traversal contains much duplicated code due to the way the source language is structured:
%
all the binary operations are separate constructors, whose subterms need to be structurally simplified before we can decide whether a rule applies.
%
The correction proof has just as much duplication because it needs to follow the structure of the call graph of the function it wants to see reduce.
%
The only saving grace here is that Coq's tactics language lets users write scripts that apply to many similar goals thus avoiding duplication in the source file.

In \Velo{}, we structured our core language's representation in an algebraic
manner so that this duplication is never needed.
%
All builtin operators (from primitive operations on builtin types to function
application itself) are represented using a single \IdrisData{Call} constructor
which takes an operation and a type-indexed list of subterms.


\ExecuteMetaData[Code/MiniCompact.idr.tex]{TermDef}

Here \IdrisType{Terms} is the pointwise lifting of \IdrisType{Term} to lists
of types. In practice we use the generic \IdrisType{All} quantifier, but this
is morally equivalent to the specialised version presented below:

\ExecuteMetaData[Code/MiniCompact.idr.tex]{AllDef}

The primitive operations can now be enumerated in a single datatype
\IdrisData{Prim} which lists the primitive operation's arguments and
the associated return type.

\begin{comment}
\IdrisData{Zero}---which takes no argument and returns a term of type \IdrisData{TyNat};
%
\IdrisData{Inc}---which takes an argument of type \IdrisData{TyNat} and returns a term
of type \IdrisData{TyNat};
%
and
%
\IdrisData{App}---which takes a function and an argument that corresponds to the type of the function's domain and returns a term that is the type of the function's co-domain.
\end{comment}

\ExecuteMetaData[Code/MiniCompact.idr.tex]{PrimDef}

Using \IdrisType{Prim}, structural operations can now be implemented by handling recursive calls on the subterms of \IdrisData{Call} nodes uniformly before dispatching on the operator to see whether additional simplifications can be deployed.
%
Similarly, all of the duplication in the correction proofs is factored out in a single place where the induction hypotheses can be invoked.


%%% Local Variables:
%%% mode: latex
%%% TeX-master: "../paper"
%%% End:
